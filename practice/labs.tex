
\documentclass[mathserif]{beamer}



%\usepackage{beamerarticle}

% ========================= GENERAL THEMES ========================= %%{{{
  
\mode<presentation> {%%%{{{

% ----- complete theme -----
\usetheme[hideothersubsections]{Hannover}

% --- inner color theme
\usecolortheme{rose}


% ----- uncover behaviour -----
\setbeamercovered{transparent}
}%%
%}}}

% get rid of navigation symbols%{{{
\setbeamertemplate{navigation symbols}{}


% ----- modify some templates of beamer -----%
\setbeamertemplate{blocks}[rounded]
%}}}
%
%}}}


% ========================= GENERAL CONFIGURATION =========================%%%{{{

% ----- misc layout -----%%{{{
\usepackage{pgfpages}
%%}}}

% ----- PGF / Tikz graphics -----%%{{{
\usepackage{tikz}
\usetikzlibrary{trees}
\usepgflibrary{arrows}
\usetikzlibrary{arrows}%%}}}

% ----- font -----%%{{{
\usepackage[spanish,english]{babel}
 \usepackage[utf8]{inputenx}%}}}

% ----- misc characters (math...) -----%%{{{
\usepackage{ccicons} % icons for Creative Commons
\usepackage{extarrows} % extra arrows
\usepackage{mathabx}   % maths fonts
\usepackage{amsmath}   % maths fonts
\usepackage{wasysym}   % maths fonts
\usepackage{stmaryrd}  % maths fonts%%}}}

% ----- extra functionalities in arrays -----%%{{{
\usepackage{array}   
\usepackage{multirow}   %%}}}

% ----- links and pdf -----%%{{{
% cf doc from hyperref for refining options
 \hypersetup{%
 colorlinks=true,    %no frame around URL
 urlcolor=olive,    % color  of URLs (brown, gray, 
}%%%}}}

% ----- R logo -----%%{{{
\newcommand{\Rlogo}{\protect\includegraphics[height=2.5ex,keepaspectratio]{Rlogo.png}}%%}}}



% ----- contents at begining of each lecture%%{{{
\AtBeginLecture{%
\frame{%
\LARGE %
\begin{center}
\structure{\insertlecture}
\end{center} } }%%}}}


%% ----- contents at begining of each section%%{{{
%  \AtBeginSection[]{%
%    \begin{frame}<beamer>{Programa}
%      \tableofcontents[currentsection, hideallsubsections]
%    \end{frame}
%  }%%%}}}


%% ----- contents at begining of each part%%{{{
% uncomment to compile the whole doc
%\AtBeginPart{%
%\frame{%
%\partpage
%} }%%}}}

%
%}}}



\begin{document}%{{{


\lecture{Introducci\'on a \Rlogo}{day1}

\begin{frame}[plain,label=d1]%%{{{
   \frametitle{?`Qu\'e es \Rlogo?}
   \begin{itemize}
      \item Programa de codigo libre
      \item Dedicado a la estad\'istica pero no solamente \ldots
      \item Entorno: programa y languaje de programaci\'on
   \end{itemize}
\end{frame}%%}}}

\begin{frame}[plain,label=d2]%%{{{
   \frametitle{?`Qu\'e es \Rlogo?}
   \begin{itemize}
      \item Funciona en modo consola
      \item Se interact\'ua escribiendo comandos 
      \item Funciones existentes
      \item Posibilidad de crear nuevas funciones
   \end{itemize}
\end{frame}%%}}}

\begin{frame}[plain,label=d3]%%{{{
   \frametitle{Preparaci\'on}
   \begin{itemize}
      \item \begin{semiverbatim} c:\\Mis Documentos\\stats \end{semiverbatim}
      \item \begin{semiverbatim} \texttt{c:\\Mis Documentos\\stats\\data} \end{semiverbatim}
      \item \begin{semiverbatim} \texttt{c:\\Mis Documentos\\stats\\docs} \end{semiverbatim}
      \item \begin{semiverbatim} \texttt{c:\\Mis Documentos\\stats\\scripts} \end{semiverbatim}
   \end{itemize}
\end{frame}%%}}}

\begin{frame}[plain,label=d4]%%{{{
   \frametitle{Empezamos}
   \begin{itemize}
      \item !`Abren R! 
      \item Prompt: \alert{$>$}
      \item \texttt{\structure{q()}}
      \item Par\'entesis, a\'un si estan vacios
      \item Consejo: !`cerrar par\'entesis inmediatamente!
   \end{itemize}
\end{frame}%%}}}

\begin{frame}[plain,label=d5]%%{{{
   \frametitle{Cerramos}
   \begin{itemize}
      \item \texttt{Save workspace image? [y/n/c]:} 
      \item Espacio de trabajo = variables que usamos en esta sesi\'on dentro de R
      \item Cerrar la ventana = N
   \end{itemize}
\end{frame}%%}}}

\begin{frame}[plain,label=d6]%%{{{
   \frametitle{Uso de scripts}
   \begin{itemize}
      \item En el espacio de trabajo, no hay los comandos que tipeamos
      \item Definir la carpeta de trabajo: 
      \item \structure{\begin{semiverbatim}setwd("c:\\Mis Documentos\\stats")\end{semiverbatim}}
      \item Para guardar comandos, se usa un script
      \item Abrir editor de texto (geany/cream/smultron/textwrangler...)
      \item Guardar el documento como:
      \item \begin{semiverbatim} c:\\Mis Documentos\\stats\\scripts\\script.R \end{semiverbatim}
   \end{itemize}
\end{frame}%%}}}

\begin{frame}[plain,label=d7]%%{{{
   \frametitle{Comentarios}
   \begin{itemize}
      \item R ignora los $\#$
      \item !`Comentarios, comentarios, comentarios!
   \end{itemize}
\end{frame}%%}}}

\begin{frame}[plain,label=d7bis]%%{{{
\frametitle{Ayuda dentro de \Rlogo}
\begin{itemize}
   \item \structure{help(q) 
   \item  \structure{help(quit)}}
   \item \structure{?q}
   \item \structure{help.start(browser="firefox")}
   \item \structure{help.search()}
   \item \structure{??quit}
\end{itemize}
\end{frame}%%}}}

\begin{frame}[plain,label=d8]%%{{{
   \frametitle{Ayuda fuera de \Rlogo}
   \framesubtitle{(!`Google es tu amigo!)}
   \begin{itemize}
      \item \url{http://cran.r-project.org/}
      \item \url{http://crantastic.org/}
      \item \url{www.rseek.org/}
      \item \url{} STOP
   \end{itemize}
\end{frame}%%}}}

\begin{frame}[plain,label=d9]%%{{{
   \frametitle{R es una calculadora}
   \begin{itemize}
      \item \structure{3+5}
      \item \structure{2*4}
      \item \structure{3-8}
      \item \structure{2.1-6*2.5}
      \item \structure{3*2-5*(2-4)/6.02}
      \item \structure{2$\hat{\;}$2}
   \end{itemize}
\end{frame}%%}}}

\begin{frame}[plain,label=d10]%%{{{
   \frametitle{R es una calculadora cientifica}
   \begin{itemize}
     \item \structure{sqrt(4)}
     \item \structure{abs(-4)}
     \item \structure{log(4)}
     \item \structure{sin(0)}
     \item \structure{exp(1)}
     \item \structure{round(3.1415,2)}
      \item  Algunas constantes: \structure{pi}
   \end{itemize}
\end{frame}%%}}}

\begin{frame}[plain,label=d11]%%{{{
   \frametitle{Creaci\'on de variables}
   \begin{itemize}
      \item Crear una variable: \structure{x $<$-- 3.14}
      \item Ver el valor de un variable: \structure{x}
      \item Usar una variable: \structure{y $<$- 2*x}
      \item Mostrar todas las variables: \structure{ls()}
      \item Borrar una variable: \structure{rm(x)}
   \end{itemize}
\end{frame}%%}}}

\begin{frame}[plain,label=d12]%%{{{
   \frametitle{Noci\'on de objetos}
   \begin{itemize}
      \item Vector: colecci\'on de elementos de mismo tipo
      \item Crear un vector: \structure{c()}
      \item \structure{x $<$-- c(1,2,3,-5,0)}
      \item \structure{y $<$-- c(``a'',``b'',``c'',``AA'')}
   \end{itemize}
\end{frame}%%}}}

\begin{frame}[plain,label=d13]%%{{{
   \frametitle{Otras maneras de crear vectores}
   \begin{itemize}
     \item \structure{rep(2.34, 5)}
     \item \structure{2:7}
     \item \structure{6:1}
     \item \structure{seq(from=1, to=2, by=0.25)}
     \item \structure{seq(from=1, to=2, length=7)}
   \end{itemize}
\end{frame}%%}}}

\begin{frame}[plain,label=d14]%%{{{
   \frametitle{Otro objetos}
   \begin{itemize}
      \item Generalizaci\'on en 2 dimensiones del vector: \structure{matrix()}
      \item Generalizaci\'on en n dimensiones del vector: \structure{array()}
      \item Para variables cualitativas: \structure{factor()}
      \item Lo m\'as com\'un con datos experimentales: \structure{data.frame()}
      \item Contenedor general de datos: \structure{list()}
   \end{itemize}
\end{frame}%%}}}

\begin{frame}[plain,label=d15]%%{{{
\frametitle{Operaciones sobre los vectores }
\begin{exampleblock}{Ejercicio:}
\begin{itemize}
   \item Crean un vector x, un vector y de mismo tama\~no que x, y un vector z de tama\~no diferente. ?`Qu\'e hacen los comandos siguientes?
\end{itemize}
   \begin{columns}[c, totalwidth=10cm]
      \begin{column}[]{4cm}
         \begin{itemize}
            \item \structure{x+1}
            \item \structure{x*2}
            \item \structure{log(x)}
            \item \structure{x$>$2}
         \end{itemize}
      \end{column}
      \begin{column}[]{4cm}
         \begin{itemize}
            \item \structure{x+y}
            \item \structure{x*y}
            \item \structure{x+z}
            \item \structure{x*z}
         \end{itemize}
      \end{column}
   \end{columns}
\end{exampleblock}
\end{frame}%%}}}

\begin{frame}[plain,label=d16]%%{{{
\frametitle{Acceso a los elementos de un vector }
\begin{itemize}
  \item Indices y operador \structure{[~]}
  \item \structure{x$<$-c(1,3,-2,0,1.3,-6.43)}
  \item \structure{x[1]}
  \item \structure{x[3]}
  \item \structure{x[-4]}
  \item \structure{x[c(1,3)]}
  \item \structure{x[c(-4,-5,-1)]}
  \item \structure{x[c(TRUE,FALSE,TRUE,TRUE,FALSE,FALSE)]}
  \item \structure{x[c(T,F)]}
  \item \structure{x[x$<$1]}
\end{itemize}
\end{frame}%%}}}

\begin{frame}[plain,label=d17]%%{{{
\frametitle{Un lenguaje para la estad\'istica}
\begin{itemize}
  \item \structure{sum(x)}
  \item \structure{length(x)}
  \item \structure{mean(x)}
  \item \structure{sd(x)}
  \item \structure{median(x)}
  \item \structure{var(x)}
  \item \structure{summary(x)}
\end{itemize}
\end{frame}%%}}}

\begin{frame}[plain,label=d18]%%{{{
\frametitle{Un languaje para la estad\'istica}
\begin{exampleblock}{Ejercicio:}
\begin{itemize}
   \item Crean un vector x de tama\~no $>10$
   \item \alert{Sin} usar \structure{mean()} ni \structure{var()}, calculan la media y la varianza de x. Guardan los resultados en las variables \emph{media.x} y \emph{varianza.x}
   \item Comparan con los resultados de \structure{mean(x)} y \structure{var(x)}
\end{itemize}
\end{exampleblock}
\end{frame}%%}}}




\begin{frame}[plain,label=d19]%%{{{
\frametitle{Importaci\'on de datos en R}
\begin{itemize}
   \item Lo m\'as flexible: \structure{read.table(~)}
   \item Variante: \structure{read.csv()}
   \item !`Necesita camino completo!
   \item Para importar datos de Excel: convertir en *.csv desde Excel y leer el archivo *.csv desde R
\end{itemize}
\end{frame}%%}}}

\begin{frame}[plain,label=d20]%%{{{
\frametitle{Pinzones de Darwin}
\begin{itemize}
   \item \structure{?read.table}
   \item \structure{picos $<$- read.table(``beaksize.csv'', header=TRUE, sep=``\textbackslash t'', dec=``,'')}
   \item Nombre del archivo
   \item El archivo tiene un membrete
   \item Separador de columnas
   \item Separador de decimales
   \item Datos del archivo en variable \emph{picos}
\end{itemize}
\end{frame}%%}}}

\begin{frame}[plain,label=d21]%%{{{
\frametitle{Objeto \emph{data.frame}}
\begin{itemize}
   \item Hoja de c\'alculo para almanecer datos
   \item Tabla de 2 dimensiones 
   \item Con nombres de columnas, nombres de l\'ineas \ldots
   \item L\'ineas = individuos
   \item Columnas = variables
\end{itemize}
\end{frame}%%}}}

\begin{frame}[plain,label=d22]%%{{{
\frametitle{Atributos de \emph{picos}}
\begin{itemize}
\item \structure{dim(picos)}
\item \structure{nrow(picos)}
\item \structure{ncol(picos)}
\item \structure{rownames(picos)}
\item \structure{colnames(picos)}
\end{itemize}
\end{frame}%%}}}

\begin{frame}[plain,label=d22bis]%%{{{
\frametitle{Acceder a las variables de \emph{picos}}
\begin{itemize}
   \item Con los indices: \structure{picos[, 1]}
   \item Con los nombres de la variables: \structure{picos[, "beaksize"]}
   \item Con el \$: \structure{picos\$beaksize}
\end{itemize}
\end{frame}%%}}}

\begin{frame}[plain,label=d23]%%{{{
\frametitle{Introducci\'on a los gr\'aficos}
\begin{itemize}
\item \structure{plot(1:10)}
\item \structure{plot(picos)}
\item Algunos par\'ametros (ver \structure{?par}): 
   \begin{description}
      \item[type =] tipo de l\'inea
      \item[main =] titulo del gr\'afico
      \item[xlab =] titulo para eje x
      \item[ylab =] titulo para eje y
      \item[col =] color
      \item[\ldots]
   \end{description}
\end{itemize}
\end{frame}%%}}}

\begin{frame}[plain,label=d24]%%{{{
\frametitle{Vector y factor}
\begin{itemize}
   \item Variable continua: vector
   \item Variable discreta: factor
   \item factor = vector + lista de valores posibles
   \item \structure{as.factor()} transforma un vector en factor
\end{itemize}
\end{frame}%%}}}

\begin{frame}[plain,label=d25]%%{{{
\frametitle{\emph{picos}: gr\'afico}
\begin{itemize}
   \item<+(1)-> \structure{plot(x=picos[,``survival''], y=picos[,``beaksize''], xlab=``Sobreviviencia'', ylab=``Largo del pico'', col=``blue'', main=``Sobreviviencia de los pinzones'')}
\end{itemize}
\begin{exampleblock}{Ejercicio}
\begin{itemize}[<+->]
   \item Chequen \structure{picos\$survival}. ?`De qu\'e tipo es?
   \item ?`La representaci\'on gr\'afica es correcta?
   \item ?`Como cambiarla?
\end{itemize}
\end{exampleblock}
\end{frame}%%}}}

\begin{frame}[plain,label=d26]%%{{{
\frametitle{\emph{picos}: gr\'afico}
\small
\begin{itemize}[<+->]
   \item \structure{plot(x=as.factor(picos[,``survival'']), y=picos[,``beaksize''], xlab=``Sobreviviencia'', ylab=``Largo del pico'', col=``blue'', main=``Sobreviviencia de los pinzones'')}
   \item Para transformar en factor de manera permanente:
   \item<+(1)-> \structure{picos[,``survival''] $<$-- as.factor(picos[,``survival''])}
\end{itemize}
\end{frame}%%}}}

\begin{frame}[plain,label=d27]%%{{{
\frametitle{M\'as sobre los gr\'aficos}
\begin{itemize}
   \item \structure{plot()} es una funci\'on gen\'erica
   \item Su comportamiento cambia en funci\'on de los datos
   \item Vean \structure{?plot} y \structure{?plot.default}
\end{itemize}
\end{frame}%%}}}

\begin{frame}[plain,plain,label=d28]%%{{{
\frametitle{Histograma}
\begin{itemize}[<+->]
   \item \structure{hist(picos[,"beaksize"])}  
   \item Par\'ametros:
      \begin{description}[<2->]
      \item[freq:] ?`N\'umeros o porcentajes?
      \item[breaks:] N\'umero de clases
      \item[plain,labels:] ?`Etiquetar las barras?
      \end{description}
   \item \structure{hist(picos[,"beaksize"], breaks=20, plain,labels=TRUE, col="lightgrey", xlab="Largo del pico", ylab="Numero de observaciones", main="Pinzones de Darwin : largo del pico") }
\end{itemize}
\end{frame}%%}}}

\begin{frame}[plain,label=d29]%%{{{
\frametitle{Un poco m\'as sobre los gr\'aficos}
\begin{itemize}
   \item \structure{example(plot)}
   \item \structure{example(plot.default)}
   \item \structure{example(hist)}
   \item \structure{demo(graphics)}
   \item \structure{demo(persp)}
   \item \structure{library(lattice); demo(lattice) }
   \item \structure{library(plotrix); demo(plotrix) }
\end{itemize}
\end{frame}%%}}}

\begin{frame}[plain,label=d30]%%{{{
\frametitle{Concurso de belleza}
\begin{exampleblock}{Ejercicio}
\begin{itemize}
   \item Trazan $f(x)=sin(x)$ con $x\in[-2\pi,2\pi]$ !`la m\'as linda posible!
\end{itemize}
\end{exampleblock}
\end{frame}%%}}}

\begin{frame}[plain,label=d31]%%{{{
\frametitle{Creaci\'on de nuevas funciones}
\begin{itemize}
   \item nombre $<$-- function(arg1,arg2,...)\{expresi\'on\}
   \item \structure{fun $<$-- function(x)\{x$\hat{~}$2\}}
   \item \structure{fun2 $<$-- function(a, b, c = 4, d = FALSE)\{\ldots\}}
   \item \structure{fun $<$-- function(x)\{y <- x$\hat{~}$2; return(y) \}}
\end{itemize}
\end{frame}%%}}}

\begin{frame}[plain,label=d32]%%{{{
\frametitle{Control de ejecuci\'on}
\begin{itemize}
\item \structure{if(condici\'on)\{ expresi\'on\}}
\item \structure{if(condici\'on)\{ expresi\'on \emph{else} \}}
\item \structure{for(variable en secuencia)\{ expresi\'on\}}
\item \structure{while(condici\'on)\{ expresi\'on\}}
\item Para m\'as informaci\'on, ver \structure{?Control}
\end{itemize}
\end{frame}%%}}}

\begin{frame}[plain,label=d33]%%{{{
\frametitle{!`Te toca a ti!}
\begin{exampleblock}{Ejercicio}
\begin{itemize}
   \item Escribir una func\'ion \emph{freq} que restituye la frecuencia de los n\'umeros dados en argumento
   \item \alert{Pista}: chequeen \structure{?sum}
\end{itemize}
\end{exampleblock}
\end{frame}%%}}}

\begin{frame}[plain,label=d34]%%{{{
\frametitle{!`Te toca a ti!}
\begin{exampleblock}{Ejercicio}
\begin{itemize}
\item Dados N=40 y M=100
   \item Usen \structure{runif()} y \structure{matrix()} para crear una matriz aleatoria de dimensiones N*M
   \item Escriben una nueva funci\'on que calcula la media de cada columna.
   \item \alert{Pista}: Hay por lo menos 3 maneras diferentes (+un bonus)
   \item \alert{Pista}: chequeen \emph{for} en \structure{?Control}, \structure{?rep}, \structure{?matmult}, \structure{?split}, \structure{?sapply}
\end{itemize}
\end{exampleblock}
\end{frame}%%}}}




\lecture{Tests estad\'isticos}{day2}

\begin{frame}[plain,label=d35]%%{{{
\frametitle{Pre-calentamiento}
\framesubtitle{Los iris de Fisher}
\begin{itemize}
   \item Cargar los datos \structure{iris.csv}
   \item Mirarlos de manera gr\'afica
   \item ?`De qu\'e tipo es el objeto \emph{iris}?  (\structure{typeof()})
   \item ?`Cu\'ales son las dimensiones de \emph{iris}?
   \item ?`Qu\'e variables contiene?
   \item ?`Cuantos individuos hay por especie?
\end{itemize}
\end{frame}%%}}}

\begin{frame}[plain,label=d36]%%{{{
\frametitle{Pre-calentamiento}
\framesubtitle{Los iris de Fisher}
\begin{itemize}
   \item Extraer sub-conjunto de datos para la especie \emph{virginica}
   \item Extraer \emph{Petal.Length} para \emph{virginica} dentro de una nueva variable llamada \emph{petal.length.vir}
   \item C\'alcular m\'inimo, m\'aximo, media y varianza de la variable Petal.Length (todas las especies de iris)
   \item \alert{Truco:} chequeen la funci\'on \structure{summary()}
\end{itemize}
\end{frame}%%}}}

\begin{frame}[plain,label=d37]%%{{{
\frametitle{Test estad\'istico}
\framesubtitle{Efecto del Rumbocur}
\begin{itemize}
   \item De acuerdo a la literatura cientifica, la prote\'ina \textsc{Prot} se encuentra en la poblaci\'on a una concentraci\'on promedio de 20 ng/ml. Se piensa que la droga \emph{Rumbocur} podr\'ia afectar el nivel de la prote\'ina en los pacientes
   \item Datos: Resultado de un experimento en el cu\'al 10 personas recibieron el \emph{Rumbocur} y 10 personas recibieron un placebo
\end{itemize}
\end{frame}%%}}}

\begin{frame}[plain,label=d38]%%{{{
\frametitle{Test estad\'istico}
\framesubtitle{Efecto del Rumbocur}
\begin{itemize}
   \item Cargar el archivo de datos \emph{rumbocur.csv}
   \item Extraer la variable \emph{drug} 
   \item Dar la hip\'otesis nula biol\'ogica
   \item Dar la hip\'otesis nula estad\'istica
   \item Se usa $\alpha=0.05$
   \item ?`Qu\'e test usar para chequear la hip\'otesis nula?
\end{itemize}
\end{frame}%%}}}

\begin{frame}[plain,label=d39]%%{{{
\frametitle{Test estad\'istico}
\framesubtitle{Efecto del Rumbocur}
\begin{itemize}
   \item Test $t$ de Student
   \medskip
   \item $$T=\frac{\mathrm{Media_{\;OBS}} -  \mathrm{Media_{\;TEO}} }{SE_{y}}$$
   \medskip
   \item $$SE_{y}=\sqrt{\frac{var(y)}{N}}$$
   \item Hacer el test ``a mano'' (usando R para calcular $T$)
\end{itemize}
\end{frame}%%}}}

\begin{frame}[plain,label=d40]%%{{{
\frametitle{Test estad\'istico}
\framesubtitle{Efecto del Rumbocur}
\begin{itemize}
   \item Buscar la funci\'on de R que permite hacer un test $t$
   \item Hacer el test con la funci\'on de R
   \item ?`Conclusi\'on estad\'istica?
   \item ?`Conclusi\'on biol\'ogica?
   \item ?`Qu\'e se necesita reportar para publicar el resultado?
\end{itemize}
\end{frame}%%}}}

\begin{frame}[plain,label=d41]%%{{{
\frametitle{Test estad\'istico}
\framesubtitle{Efecto del Rumbocur}
\begin{itemize}[<+-| visible@+->]
   \item ?`Qu\'e se necesita para chequear que la diferencia en concentraci\'on de \emph{Prot} se debe al Rumbocur?
   \item !`Se necesita un control!
   \item Para comparar el grupo que recibi\'o la droga con el grupo que recibi\'o el placebo, ?`qu\'e test se va a usar?
   \item Dar la hip\'otesis nula biol\'ogica del nuevo test
   \item Dar la hip\'otesis nula estad\'istica
\end{itemize}
\end{frame}%%}}}

\begin{frame}[plain,label=d42]%%{{{
\frametitle{Test estad\'istico}
\framesubtitle{Efecto del Rumbocur}
\begin{itemize}
   \item Correr el test (directamente con la funci\'on de R)
   \item ?`Conclusi\'on estad\'istica?
   \item ?`Conclusi\'on biol\'ogica?
   \item ?`Qu\'e se necesita reportar para publicar el resultado?
\end{itemize}
\end{frame}%%}}}

\begin{frame}[plain,label=d43]%%{{{
\frametitle{Observaci\'on}
\begin{itemize}
   \item Equivalente no par\'ametrico del test t: test de Mann-Whitney (ver \structure{?wilcox.test})
   \item Si el mismo grupo recibe el placebo por 2 semanas y despu\'es recibe el Rumbocur por 2 semanas, las muestras estan asociadas. Se necesite usar la versi\'on ``asociada'' del test t (argumento \structure{paired=T} en \structure{t.test()})
\end{itemize}
\end{frame}%%}}}

\begin{frame}[plain,label=d44]%%{{{
\frametitle{Test $\chi^2$ de ajuste}
\framesubtitle{con $k>2$ muestras}
\begin{itemize}
   \item Una empresa crea una nueva bebida. Hay 3 versiones A, B, y C. 120 personas prueban las 3 versiones y deciden cual prefieren. 30 personas prefieren A, 54 prefieren B, y 36 personas prefieren C.
   \item  Crear el vector de datos
   \item ?`Hip\'otesis nula biol\'ogica?
   \item ?`Hip\'otesis nula estad\'istica?
   \item Correr el test y concluir
\end{itemize}
\end{frame}%%}}}

\begin{frame}[plain,label=d45]%%{{{
\frametitle{Test $\chi^2$ de independencia}
\framesubtitle{Tabla de contingencia}
\begin{itemize}
   \item Cargar los datos \emph{wasp.csv}
   \item ?`Cua\'ales son la hip\'otesis?
   \item Transformarlo en una tabla de contingencia (ver \structure{?xtabs})
   \item Correr el test y concluir
\end{itemize}
\end{frame}%%}}}





\lecture{Regresi\'on lineal}{day3}



\begin{frame}[plain,label=d46]%%{{{
\frametitle{Pre-calentamiento}
\begin{itemize}
   \item Hagan el vector siguiente con 1 comando
  %\item $0\; 0\; 0\; -1\; -1\; -1\; -1\; -1\; -1\; -1\; -1\; -1\; -1\; 2\; 2\; 2\; 2\; 2$
  \item $0\;0\;0-1-1-1-1-1-1-1-1-1-1\;2\;2\;2\;2\;2$
   \item Truco: vean \structure{?rep()} y piensen en vectores
\end{itemize}
\end{frame}%%}}}

\begin{frame}[plain,label=d47]%%{{{
\frametitle{Correlaci\'on}
\begin{itemize}
   \item Cargar los datos \emph{cars} con \structure{data(cars)}
   \item Estos datos est\'an en pies y millas por hora \ldots
   \item Convertir en el sistema m\'etrico (1pie=0.3048m y 1Mi/h=1.609344 km/h)
   \item Hacer la representaci\'on gr\'afica con titulos en los ejes
   \item Viendo el gr\'afico, que se puede decir?
   \item Calcular el coeficiente de correlaci\'on
   \item ?`Qu\'e se puede concluir respecto a la fuerza de la relaci\'on lineal entre las dos variables?
\end{itemize}
\end{frame}%%}}}

\begin{frame}[plain,label=d48]%%{{{
\frametitle{Fumadores y tensi\'on arterial}
\begin{itemize}
   \item En una poblaci\'on, se toma una muestra aleatoria de 34 personas (17 fumadores, 17 no fumadores) a quien se pregunt\'o la edad y se midi\'o la tensi\'on arterial (en mmHg)
   \item Cargar los datos desde \emph{epidemio.txt}
   \item Hacer la representaci\'on gr\'afica de la tensi\'on arterial con respecto a la edad
\end{itemize}
\end{frame}%%}}}

\begin{frame}[plain,label=d49]%%{{{
\frametitle{Fumadores y tensi\'on arterial}
\begin{itemize}
   \item ?`Qu\'e se puede decir?
   \item ?`Como se puede chequear?
   \item Ajustar el modelo lineal (\textbf{L}inear \textbf{M}odel)
   \item Tension = a*Edad + b
   \item Trazar la l\'inea de regresi\'on
\end{itemize}
\end{frame}%%}}}

\begin{frame}[plain,label=d50]%%{{{
\frametitle{Fumadores y tensi\'on arterial}
\begin{itemize}
   \item Chequear \structure{?lm}
   \item Formula: Tension $\sim$ Edad
   \item \structure{?coefficients}
   \item \structure{?abline}
   \item Probar \structure{names(m1)}
   \item \structure{m1\$coefficients}
   \item \structure{summary(m1)}
\end{itemize}
\end{frame}%%}}}

\begin{frame}[plain,label=d52]%%{{{
\frametitle{Fumadores y tensi\'on arterial}
\begin{semiverbatim}
\begin{itemize}
   \item[$>$] epidemio <- read.table(data/epidemio.txt, header=T, skip=1)
   \item[$>$] plot(tension $\sim$ edad, data=epidemio)
   \item[$>$] m1 <- lm(tension $\sim$ edad, data=epidemio)
   \item[$>$] summary(modelo)
   \item[$>$] abline(modelo)
\end{itemize}
\end{semiverbatim}
\end{frame}%%}}}

\begin{frame}[plain,label=d53]%%{{{
\frametitle{}
\begin{itemize}
   \item 
   \item 
   \item 
\end{itemize}
\end{frame}%%}}}



\end{document}
%}}}













